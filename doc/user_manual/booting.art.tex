\section{flash命令}
~flash~命令列表
~flash命令用法~: flash $<$子命令$>$ [可选参数 $<$值$>$]
\begin{table}[htbp]
\setlength{\parindent}{0pt}
\begin{tabular}{|c|c|} \hline
\small子命令名称 & \small命令说明 \\
\hline
  dump &   查看~flash~页内容,以~page~为单位,包括~oob~ \\
\hline
  erase &   以~block~为单位,擦除~flash~一块内容 \\
\hline
 partshow &  查看~Flash~分区信息 \\
\hline
  parterase &   以~分区~为单位,擦除一个分区 \\
\hline
  scanbb &   扫描指定分区上的坏快 \\
\hline
  load &   从~Flash~中加载数据到~DDR~ \\
\hline
\end{tabular}
\end{table}

\begin{enumerate}
\item 命令名称: dump\\
参数介绍:
\begin{table}[htbp]
\setlength{\parindent}{0pt}
\begin{tabular}{|c|c|}
\hline
 -b &  读取指定flash块号,查看块内第一页内容	\\ \hline
 -p &  读取指定flash页号,查看一页内容\\ \hline
 -a &  设定读取的起始地址。\\ \hline
\end{tabular}
\end{table}
\pagebreak[4]

\item 命令名称:erase\\
参数介绍:
\begin{table}[htbp]
\setlength{\parindent}{0pt}
\begin{tabular}{|c|c|}
\hline
  -a &    指定擦除的起始地址(以字节为单位)\\ \hline
  -b &   指定擦除的开始块号(以块为单位) \\ \hline
  -d &   擦除时忽略坏块 \\ \hline
  -m &    擦除的同时写入cleanmark\\ \hline
  -l &   指定擦除长度 \\ \hline
\end{tabular}
\end{table}

\item 命令名称:parterase \\
参数介绍:
\begin{table}[htbp]
\centering
\begin{tabular}{|c|c|}
\hline
 -p &   指定擦除分区的区号 \\ \hline
 -d &    擦除时忽略坏块\\ \hline
\end{tabular}
\end{table}

\item 命令名称:load \\
参数说明:
\begin{tabular}{|c|l|}
\hline
 -b &  读取指定~flash~块号,查看块内第一页内容	\\ \hline
 -p &  读取指定~flash~页号,查看一页内容\\ \hline
 -a &  设定读取的起始地址。\\ \hline
 -m & 写入~mem~的起始址址。\\ \hline
 -s & 从~flash~中读取的字节数\\ \hline
\end{tabular}
\end{enumerate}

\section{网络命令}

\noindent网络命令列表
\begin{table}[htbp]
\setlength{\parindent}{0pt}
\begin{tabular}{|c|c|}
\hline
\small命令名称 & \small命令说明 \\
\hline
 ping &   用来测试板子和主机是否连通 \\
\hline
  ifconfig&  设置开发板的IP和MAC,并制定和开发板相连的主机IP \\
\hline
  tftp&   通过tftp协议下载文件 \\
\hline
\end{tabular}
\end{table}

\noindent命令名称:ping\\
参数介绍:暂无
\begin{table}[htbp]
\setlength{\parindent}{0pt}
\begin{tabular}{|c|c|}
\end{tabular}
\end{table}
\pagebreak[4]

\noindent命令名称:ifconfig\\
参数介绍:
\begin{table}[htbp]
\setlength{\parindent}{0pt}
\begin{tabular}{|c|c|}
\hline
 -l &   配置本地~IP~ \\
\hline
 -s &   配置服务器~IP~ \\
\hline
  -m&   配置~MAC~地址 \\
\hline
\end{tabular}
\end{table}

\noindent命令名称:tftp\\
参数介绍:
\begin{table}[htbp]
\setlength{\parindent}{0pt}
\begin{tabular}{|c|c|}
\hline
 -f &   指定下载的文件名 \\
\hline
 -s &   设定服务端~IP~ \\
\hline
 -m &   下载的内容放在内存里,即,不烧录到~flash~上 \\
\hline
\end{tabular}
\end{table}
\pagebreak[4]

\section{boot~命令}
\begin{table}[htbp]
\setlength{\parindent}{0pt}
\begin{tabular}{|c|c|}
\hline
 命令名称 & 命令说明\\
\hline
~boot~ & 引导操作系统\\
\hline
\end{tabular}
\end{table}

\noindent命令名称:boot\\
参数介绍:\\
\begin{table}[htbp]
\setlength{\parindent}{0pt}
\begin{tabular}{|p{2.5cm}|p{8.5cm}|}
\hline
-~t~ [~filename~] &若指定~filename~,则通过~tftp~下载~kernel~ ~image~文件;否则从本地的~linux~分区下载~kernel~ ~image~文件 \\ \hline
-~r~ [~filename~] &用~ramdisk~启动。指定~filename~,则通过~tftp~下载~ramdisk~ ~image~;否则从本地~ramdisk~分区下载。 \\ \hline
-~f~ [~N~] & 指定~rootfs~分区,~N~为分区号 \\ \hline
-~n~ [~ip~:~path~] & 用~nfs~方式~mount~ ~rootfs~ \\ \hline
-~v~ &仅显示~kernel~启动参数,但并不真正引导~OS~ \\ \hline
\end{tabular}
\end{table}
\pagebreak[4]

\section{memory~命令}
\begin{figure}[H]
\centering
\begin{tabular}{|c|c|}
\hline
命令名称 & 命令说明\\ \hline
md & 显示~memory~数据 \\ \hline
mw & 将数据写入~memory~ \\ \hline
memset & 将某个~memory~空间写入值。\\ \hline
\end{tabular}
\caption{md}
\end{figure}

\section{其他命令}
\begin{table}[htbp]
\setlength{\parindent}{0pt}
\begin{tabular}{|c|c|}
\hline
命令名称 & 命令说明\\
\hline
~confreset~ & 恢复~g-bios~默认设置\\ \hline
~conflist~ & ~列出~g-bios~当前配置信息\\ \hline
~kermit~ &通过串口将指定的文件烧录到当前~Flash~分区 \\ \hline
~ls~ & 查看分区信息\\ \hline
~cd~ & 切换分区\\ \hline
~go~ & 跳到指定位置执行程序\\ \hline
\end{tabular}
\end{table}

\noindent命令名称:~confreset~\\
参数介绍:无\\
\\
\noindent命令名称:~kermit~\\
参数介绍:无\\
\\
\noindent命令名称:~ls~\\
参数介绍:无\\
\\
\noindent命令名称:~cd~\\
参数介绍:\\

\noindent 命令名称:~go~ \\
参数介绍:<mem\_add> 即将跳转的~memory address~,mem\_add~可十进制表示,也可十六进制表示。
\\eg: go 0xc000000 跳转到~0xc000000~处执行。
\begin{table}[htbp]
\setlength{\parindent}{0pt}
\begin{tabular}{|c|c|}
\hline
[~N~] & 目标区号\\
\hline
\end{tabular}
\end{table}

\chapter{Loading Linux}

%\section{Add New Commands}
%g-bios~添加命令规范。
%\begin{lstlisting}
%static char app_option[][CMD_OPTION_LEN] = {};
%
%#include <getopt.h>
%
%int getopt(int argc, char *argv[], const char *optstring, char **arg);
%\end{lstlisting}

boot command design ..

\section{TFTP + NFS}

\indent 其中~NFS~服务配置和编译~linux kernel~部分详情请参阅$<<$MaxWit Lablin开发者手册$>>$第一卷\\
\indent 在~g-bios~命令行下,输入:\\

\begin{verbatim}
g-bios: 0# boot -t zImage -n 192.168.0.2:/home/maxwit/maxwit/rootfs
【说明】
-t [filename]:用tftp方式下载指定的kernel image
-n [nfs_server:/nfs/path/]: 用NFS方式mout rootfs。也可以加上参数,如:-n 192.168.0.111:/path/to/nfs
\end{verbatim}

~boot~程序具有记录功能,即,能记住用户输入的参数,换句话,再次输入~boot~时不再需要输入参数了,除非你想重设参数。

\section{FLASH + NFS}
\begin{verbatim}
g-bios: 1# cd 3  (进入Linux分区)
g-bios: 3# ls (列出当前分区信息)
        Partition Type = "linux"
        Partition Base = 0x00080000 (512K)
        Partition Size = 0x00200000 (2M)
        Host Device    = NAND 256MB 3.3V 8-bit
        MTD Deivce     = /dev/mtdblock3
        Image File     = "zImage" (1968220 bytes)
g-bios: 3# tftp zImage (下载zImage 到当前分区)
 "zImage": 192.168.2.101 => 192.168.2.100
 1968220(1M898K92B) loaded

g-bios: 1# boot  -t  -n 192.168.2.11:/root/maxwit/rootfs
【说明】
-t  不加参数,从Linux分区Load kernel image
-n [nfs_server:/nfs/path/]: 用NFS方式mount rootfs。也可以加上参数。如:-n 192.168.0.111:/home/maxwit/maxwit/rootfs。
\end{verbatim}

\section{Booting from Flash}

\begin{verbatim}
g-bios: 1# cd 3  (进入Linux分区)
g-bios: 3# ls (列出当前分区信息)
        Partition Type = "linux"
        Partition Base = 0x00080000 (512K)
        Partition Size = 0x00200000 (2M)
        Host Device    = NAND 256MB 3.3V 8-bit
        MTD Deivce     = /dev/mtdblock3
        Image File     = "zImage" (1968220 bytes)
g-bios: 3# tftp zImage (下载zImage 到当前分区)
 "zImage": 192.168.2.101 => 192.168.2.100
 1968220(1M898K92B) loaded

g-bios: 3# cd 5 (进入Rootfs分区)
g-bios: 5# tftp rootfs_l.jffs2 (下载zImage 到当前分区)
g-bios: 5# boot -t -f 5
【说明】
-t :不加参数,从Linux分区Load kernel image
-f [N]:指定rootfs的分区,N为分区号
\end{verbatim}
